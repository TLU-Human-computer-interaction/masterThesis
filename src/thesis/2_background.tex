% !TeX root = thesis.tex
\documentclass{master_thesis}
\addbibresource{refs.bib}

\begin{document}

\section{Background}

\subfile{2accessibility.tex}

\subsection{Component libraries}

Component Driven Development (CDD) is a development methodology that anchors the build process around components \citep{Coleman2017}. In \citeyear{Coleman2017} \citeauthor{Coleman2017} called CDD the biggest trend in user interface (UI) development.

A component is a well-defined and independent piece of UI, like a button, checkbox or card. This approach correlates well with other similar widespread principles that promote modularity in software development like atomic design and micro-frontends. Designing each component as a standalone unit improves maintenance, reusability, testing, shortens the learning curve and makes development faster.\citep{Ella2019}

The key to success here is separation of concerns and isolating a logical piece so that it can be worked without distractions. It promotes concentrating on details and refining the element as much as possible. These pieces can be combined into more complex component if needed and when combined they can make up views that become the whole site or webpage.

A component explorer is a tool that is often used in this approach. Working on a single component by manipulating the entire webpage or app to a certain state can be difficult. Component explorer is a separate application that showcases the components in various states. This allows developers to test a given component in all the states that have been defined in isolation and makes it easy to build one component at a time. It also makes it easier to go through all the possible states in one component and promotes reusability of these elements. \citep{Coleman2017}

Bit, Storybook and Styleguidist are popular tools used in component library development. Bit lets you pack the bundled and encapsulated components and share them to the cloud where your team can visually explore them. Storybook supports multiple frameworks and provides a rapid component development and test environment. The environment also allows you to present and document your library for better reusability. Styleguidist is useful for documentation and demoing different components. \citep{Ella2019}

A collection of multiple components that is shared to be reused is called a component library. Design systems include values and principles along, branding, guidelines  and all the building blocks and design patterns to create a successful product or service. It governs the design process in the organization. A component library or UI kit, UI library, UI component library is one of the building blocks of a design system. \citep{Ramotion2022}

Most big companies have a design system that includes a component library. Often these might be published with an open source code, making it possible to reuse and extend on them. Some of the more known ones include Material-UI \citep{MUS}, Adobe Spectrum \citep{Adobe} and Bootstrap \citep{Collings}. They help keep things consistent across multiple projects by providing small building blocks that can be used to create complex designs.  I can be built with HTML or by using a framework like React or Vue.

Using a component library provides consistency and better quality. These components can be polished over time to provide the best user experience to end users when they are integrated to the final product. The effort that can be put into developing a component that will be used in multiple projects is bigger than what would be reasonable for a single use case.

\end{document}
