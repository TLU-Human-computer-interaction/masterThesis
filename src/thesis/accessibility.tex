\documentclass{master_thesis_section}

\begin{document}

\section{What is accessibility?}

\subsection{What is disability?}
WHO estimates that 1.3 billion people – or about 16\% of the global population – experience a significant disability and this number is growing. Disability is a part of being human and persons with disabilities are different from each other. \citep{WHODisability2022} \\
% Some more general accessibility principles not related ot web a11y
Disability beyond the 1.3 billion reported – being old and maybe just not at a 100\%. \citep{}

\subsection{Universal design}
\subsection{Web accessibility}

"The power of the Web is in its universality. Access by everyone regardless of disability is an essential aspect," said Tim Berners-Lee, W3C Director and inventor of the World Wide Web \citep{worldwidewebconsortium1997}.

Essentially accessibility in providing equal access to goods and services to everyone regardless or their age, gender or disabilities. These principles can be applied to any various fields. For example in architecture it could be designing buildings in a way that that can be accessed by people who can walk on their own as well as the ones who need to use a wheelchair. In digital products it is more often related to the senses we use to consume content. Everything could be equally accessible whether I want to see or hear it for example.

Web accessibility is standard practices and requirements are defined in WCAG. People who consume web content have different abilities, because of their experience or different conditions that might affect their physical or mental capabilities. Someone might break their hand and be temporarily unable to use a mouse to navigate, or they might have been born blind and rely entirely on assistive technologies to consume content. The same content should be equally accessible for them both.
%TODO some history how did accessibility get started what is the history behind web accessibility?

\end{document}

