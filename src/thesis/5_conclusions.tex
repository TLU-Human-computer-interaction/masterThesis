% !TeX root = thesis.tex
\documentclass{master_thesis}
\addbibresource{refs.bib}

\begin{document}

\section{Discussion}

\subsection{Limitations of the study}

The tool that was used was chosen with the limitation of our current tooling. There are plenty of other options out there. Some free some paid and some of them might be more effective than the one we decided to use.

We had limited time. The tool was in use for a bit more than 5 months. It will continue to be used and it would be interesting to see how well it performs long term.

Manual accessibility audit might have not had enough highly qualified accessibility experts on the team.

\subsection{Future research}

Storybook 6 was used in Pipedrive's component library during the case study, but it seems that the next major version of this component exploer provides improved tools for testing. At the time of the study some of these possibilities where tested to get and overview of what might be possible. The new major verison has a feature to add play-functions or interactions to each story (example). These interactions are then run as unit tests. The asseccibility add-on works the same way and it it important to note that these accessibility tests will be run after the interaction finishes. This means that combinging the add-on with interactions would allow to test some states that where hidden before. For example it would be possible to add an user interaction to press a button for the 11 examples that had a button trigger and where the real component did not get tested. This scenario was tried out in the example component library and it works.

Another added benefit that Storybook 7 provides is a test-runner is meant for running the initeractions that can be added to each story as unit tests. Accessibility test can be easily added to this globally without needing to do any extra steps in every component story. This will make it possible to run the same accessibility tests that can be seen in the add-ons panel in \ac{cli} and to \ac{ci} testing phase.

\todo{Adding accessibility tests from the beginning - would that have a bigger impact? }
\todo{Possibilities in SB 7 could be explored, test runner would even alllow to use a different accessibility testing engine}
\todo{AI for accessibility evaluation}

\end{document}