\documentclass{master_thesis}

%  This needs to be here so that IntelliSense works for citations

\addbibresource{refs.bib}

\begin{document}


%===BEGIN TITLE PAGE
\thispagestyle{empty}
\begin{center}

\large
\iflanguage{english}{%
Tallinn University\\
% School of Digital Technologies\\
MSc Human-Computer Interaction\\
}{%\iflanguage
Tallinna Ülikool\\
% Digitehnoloogiate instituut\\
Inimese ja arvuti interaktsioon\\
}%\iflanguage

\vspace*{\stretch{3}}

% title of the thesis must be capitalized like this https://capitalizemytitle.com/
\huge \textbf{
	\iflanguage{english}{Improving Web Accessibility Through Continuous Evaluation of Component Libraries}
	{Veebi Ligipääsetavuse Parendamine Läbi Komponentiteekide Pideva Testimise}}

\vspace{10mm}

\Large
\iflanguage{english}{Master’s Thesis}{Magistritöö} \\
\iflanguage{english}{Author}{Autor}: Mariann Tapfer \\

\end{center}

\vspace{10mm}

\begin{flushright}
 {
 \setlength{\extrarowheight}{5pt}
 \begin{tabular}{r l}
   \iflanguage{english}{Supervisors}{Juhendaja(d)}: &
   Mustafa Can Özdemir, MSc \\
    & Mari-Ell Mets, \acs{iaapwas}
 \end{tabular}
 }
\end{flushright}

\vspace*{\stretch{3}}

\vfill
\centerline{\large Tallinn \the\year}

%===END TITLE PAGE

% Remove this before you submit a final paper
\listoftodos[Notes]

\section*{Abstract}

% The abstract is a short summary of the dissertation that comes at the beginning of the paper. It outlines all the major points your paper discusses and often mentions the methodology briefly. Abstracts should be only one paragraph, about 300 to 500 words.

% The term abstract is often used interchangeably with executive summary. While common usage suggests they’re the same, they’re technically different: An executive summary discusses the findings or conclusion of the research, whereas an abstract does not.

\section*{Acknowledgements}

\tableofcontents

% The table of contents lists all titles for chapters, headings, and subheadings, as well as their corresponding page numbers. Moreover, the table of contents also includes the supplementary sections—such as the bibliography, appendices, and optional sections like a glossary, list of abbreviations, or a list of figures and tables.

\listoffigures
% \addcontentsline{toc}{section}{List of Figures}

\listoftables
% \addcontentsline{toc}{section}{List of Tables}

\printacronyms[name=List of Abbreviation]

\subfile{1_introduction.tex}

\subfile{2_background.tex}

\subfile{3_research.tex}

\subfile{4_results.tex}

\subfile{5_conclusions.tex}

% With this all references in ref.bib file are listed. without only the ones that are actually used
% \nocite{*}
\emergencystretch=1em
\printbibliography[heading=bibintoc]

\subfile{appendices.tex}

\end{document}

