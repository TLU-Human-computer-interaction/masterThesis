\documentclass{master_thesis}
\addbibresource{refs.bib}


\begin{document}
\subsection{Automated testing}
Continuous integration or CI for short  is a software development practice where members of the team integrate that work frequently and each integration is verified with and automated build that includes test to detect and errors as quickly as possible. This is believed to help develop high quality software more rapidly. One of the practices in CI is making your code Self-Testing - adding a suite of automated tests that can check a large part of the code base for bugs. These tests need to be easy to trigger and indicate of any failures.
\citep{Fowler2006}

This last principle could be applied to accessibility. It goes well modern software development principles. Test will not catch everything, but they will catch enough to make it worthwhile. After the initial setup they should not need a lot of maintenance and will be run every time someone contributes to the codebase. This can act as a very effective gatekeeper for any potential accessibility issues.

\subsubsection{How does it work?}

Web content is essentially code and any code can be analyzed and compared against rule sets. Accessibility testing can be done by experts, real users or by code. The last one is what we call automated testing.

It can be set up in different way - as a browser plugin that check the site you are currently on against a rule set or a test written to ensure that a piece of code does not violate a certain rule or unit test that will ensure that the sites' compliance with a set of rules is consistent.

These kinds of test can detect up to 60\% of violations. \todo{Find citation}
There is a limit to what can be detected. More testing will always be required to ensure that the content is fully accessible. These kinds of tests can be set up to run automatically, and they provide measurable results. This means it a good way to monitor compliance with WCAG rules consistently without any extra effort. This can help avoid unwanted changes and show easy fixes in  your code.

WCAG 2.0 was made in a way that works better for automated testing. \todo{Find citation}

\subsubsection{Types of tools}

\citep{Sane2021}
\citep{AbouZahra2017}
\citep{AbouZahra2020}
\citep{Alsaeedi2020}
\citep{WebAIM2019}
What are the most common methods for testing?

\begin{itemize}
	\item Testing with users
	\item Testing using experts
	\item Testing by using checkers that run in a browser
\end{itemize}
\subsubsection{Automated accessibility testing.}

\begin{itemize}
	\item How can accessibility testing be automated?
	\item In software development CI/CD (Continuous integration / Continuous delivery) is commonly used. How can we use the same principle for automated testing. Look et this \citep{}
	\item What are common strategies used by other companies?
\end{itemize}

\subsubsection{Limitations of automated tests}

The automated accessibility checkers that are available now will not catch all the errors. They work on certain types of issues very well and not so well on others. The use of AI in could make these tools even more effective, \todo{Citation} but until then they should always be combined with testing with users or experts to ensure that the products that you are developing is truly accessible.

\end{document}

