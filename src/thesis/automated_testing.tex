\documentclass{master_thesis_section}

\begin{document}

\section{Automated accessibility testing}

Web content is essentially code and any code can be analyzed and compared against rule sets. Accessibility testing can be done by experts, real users or by code. The last one is what we call automated testing. \\
It can be set up in different way - as a browser plugin that check the site you are currently on against a rule set or a test written to ensure that a piece of code does not violate a certain rule or unit test that will ensure that the sites' compliance with a set of rules is consistent. \\
These kinds of test can detect up to 60\% of violations.
% citation needed
There is a limit to what can be detected. More testing will always be required to ensure that the content is fully accessible. These kinds of tests can be set up to run automatically and they provide measurable results. This means it a good way to monitor compliance with WCAG rules consistently without any extra effort. This can help avoid unwanted changes and show easy fixes in  your code.

\section{Common ways of testing for accessibility?}

\citep{Sane2021}
\citep{InitiativeWAIa}
\citep{WebAIM2019}

What are the most common methods for testing?

\begin{itemize}
	\item Testing with users
	\item Testing using experts
	\item Testing by using checkers that run in a browser
\end{itemize}

\section{Automated accessibility testing.}

\begin{itemize}
	\item How can accessibility testing be automated?
	\item In software development CI/CD (Continuous integration / Continuous delivery) is commonly used. How can we use the same principle for automated testing. Look et this \citep{}
	\item What are common strategies used by other companies?
\end{itemize}

\end{document}

