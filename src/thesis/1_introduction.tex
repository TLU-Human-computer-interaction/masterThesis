% !TeX root = thesis.tex
\documentclass{master_thesis}
\addbibresource{refs.bib}
\begin{document}
\section{Introduction} \label{chap:intro}
\info{Overview of the entire thesis.  It ends by describing the organization of this document.}

% Provide preliminary background information that puts your research in context
According to Internet World Stats 67.9\% of the global world population uses the internet \citep{MMG2023} and a big part of these people might face difficulties accessing the content that is provided on the web. Web accessibility focuses on making the web equally accessible to everyone.

% Clarify the focus of your study

The most common standard to follow for making webpages accessible is \ac{wcag} and there are different methods and tools for testing the compliance of web content with it. One of them is automated testing. The goal of this research is to understand the limitations of automated accessibility testing in the context of evaluating the accessibility of a component library.

% Point out the value of your research(including secondary research)

To explore this a case study and manual accessibility audit will be carried out in Pipedrive's component library. This will give valuable insights into what using automated accessibility testing tools would look like in a real-life situation. The aim of this is to understand the pros and cons of these tools.

% Specify your specific research aims and objectives

The \nameref{chap:intro} chapter will outline the research problem and goal and will end with phrasing the research questions that this thesis will aim to answer. The \nameref{chap:background} chapter will put this research into context. Looking into what is web accessibility and what are the common standards and regulations for it, what are the different methods and tools for evaluating it. The case study focuses on accessibility evaluations in a component library so the last part of this chapter will explain the concept, usage and importance of component libraries in web development.

\nameref{chap:research} chapter will start by going over the methodology for this research and continue with explaining how the 3 main parts of this work: a survey conducted in Pipedrive to get insights about the current state of knowledge and awareness around accessibility, an explorative case study about the usage of automated accessibility testing conducted in Pipedrive's component library and overview of a manual accessibility audit carried out in the same component library.

\nameref{chap:results} chapter will analyze the results from the survey, case study and manual audit to understand the capabilities and limitations of automated accessibility testing tools. The last chapter, \nameref{chap:conclusions} will focus on discussing the limitations of this research as well as exploring possibilities of future research on similar or related subjects.

\subsection{Research Problem and Goal}

\info{ Research Problem and Significance: What is the problem you are addressing? Why is this a problem? Why is this important to HCI? Supported scientific references? ate 3 4 ½ page max \\
Research Goal and Motivation: Why are you doing this research? what will be your contributions? Supported scientific references? ate 3 4 ½ page max}

According to \ac{who} 1 billion people or about 16\% of the world's population are estimated to have a significant disability \citep{WHO2022}. They may not be able to use websites and mobile applications in a conventional way. To provide equal possibilities, digital environments need to be made accessible. Focusing on accessibility also has a profound effect on the overall quality of products and services and therefore provides a big business value \citep{Miesenberger2020}.

Developers find it hard to maintain a high level of accessibility over time as it is considered more at the first release of the product \citep{Paterno2020}. \ac{ci} is used to speed up development and maintain general code quality \citep{Zhao2017} and it could also help in maintaining a high level of accessibility by testing the added code against predefined accessibility requirements. There is some initial research into the current state of accessibility testing in \ac{ci}, but it could be improved by doing a case study for an organization \citep{Sane2021, KelseyAdkins2022}.

Design Systems are an increasingly common way to build websites because they help in producing a consistent user experience \citep{Yew2020}. Influencing how these are built is important in the field of HCI because it will allow us to play a role in shaping the future of User Interfaces. Many companies have design systems to help maintain consistency both in what the end users experience and how the code is written. This is a good place to start with establishing a basic level of accessibility. Making changes there will have the widest impact and will build a solid foundation.

Most accessibility checkers are intended to be used for the whole page or website. In the case study conducted during this research automatic accessibility tests will be run on the individual components in a component library because that's a key part of most Design Systems \citep{Yew2020}. Some aspects of accessibility, like correct usage of page headers or page language, can only be evaluated on the full page. Parallels can be drawn between isolated component testing and unit tests in software development. The goal of unit tests is to verify that small pieces of code function properly \citep[p.60]{Humble2010}. Similarly, component testing should ensure that this specific component does not have any issues related to accessibility.

The next step would be running end-to-end tests on the whole page or application that is made up of these small parts. End-to-end tests should imitate what would happen when end users are using the product. When individual component tests are working well then there should already be fewer issues when you test the whole page.

This thesis will explore the potential pros and cons of accessibility testing in the \ac{ci} of a component library by trying out an automated accessibility evaluation tool in Pipedrive's component library.

\subsection{Research Questions}

	\begin{RQlist}
		\item How good is the knowledge about accessibility standards, tools and best practices in the company before integrating the accessibility testing tool?
		\item What kind of errors can be caught by running automated accessibility tests on a component library?
		\item To what extent can integrating automated testing into a component library's development pipeline help improve its compliance with \ac{wcag}?
		% This might be the most difficult question to answer. There is not enough time to do another manual audit, but I can generate a new automatic report. What can I read into that? If the state is better now is it because of the automated accessibility tool integration or because we also did a manual audit? Maybe I can get some insights from the end questionnaire + reaching out to some devs who have used these tools during that period. Is this a one-time fix and if now then what evidence do I have that it would have a wider impact?
		\item What are the biggest problems of integrating automated accessibility testing into a component library's development workflow?
	\end{RQlist}

\end{document}
