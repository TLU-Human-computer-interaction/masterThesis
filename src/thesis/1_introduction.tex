% !TeX root = thesis.tex
\documentclass{master_thesis}
\addbibresource{refs.bib}
\begin{document}
\section{Introduction} \label{chap:intro}
% \info{Overview of the entire thesis.  It ends by describing the organization of this document.}

% The first of the “core chapters” and the de facto beginning of your paper, your introduction sets up your research topic and provides the necessary background context to understand it. Here, you plainly state your thesis statement or research question and give a glimpse of how your paper discusses it.

% The introduction is typically structured with each chapter getting its own brief summary. It should hint at your methodology and outline your approach (without going into too much detail), as well as explain the current state of the topic’s research so the reader knows where your dissertation fits in.

% How long should a dissertation introduction be? The unofficial rule is 10 percent of the entire paper, so if your dissertation is 20,000 words, your introduction should be about 2,000 words. Keep in mind this is a rough estimate, as your introduction could vary.

% Provide preliminary background information that puts your research in context
According to Internet World Stats 67.9\% of the global world population uses the internet \citep{MMG2023} and a big part of these people might face difficulties accessing the content that is provided on the web. Web accessibility focuses on making the web equally accessible to everyone.

% Clarify the focus of your study

The most commonly used standard for making webpages accessible is the Web Content Accessibility Guidelines \ac{wcag}, and there are various methods and tools available for testing web content's compliance with it. One such method is automated testing. The aim of this research is to investigate the possibilities and limitations of automated accessibility testing when evaluating a component library's compliance with common accessibility standards.

% Point out the value of your research(including secondary research)

% To explore this, a case study and a manual accessibility audit will be conducted in the author's current workplace, Pipedrive. Pipedrive is a software company building a web-based sales \ac{crm}. The case study will involve integrating an automated accessibility testing tool into the company's component library and observing its usage. This will provide valuable insights into the practical implications of using automated accessibility testing tools in a real-world setting. The goal is to understand the advantages and disadvantages of these tools.

To investigate this, a case study and manual accessibility audit will be conducted in Pipedrive, a software company that develops a web-based sales \ac{crm} and the workplace of the author of this thesis. The case study will involve integrating an automated accessibility testing tool into the company's component library and observing its usage. This will provide valuable insights into the practical implications of using automated accessibility testing tools in a real-world setting.
% The aim is to understand the advantages and disadvantages of these tools.

% Specify your specific research aims and objectives

The "\nameref{chap:intro}" chapter will outline the research problem and objectives, and conclude by stating the research questions that this thesis aims to answer. The subsequent chapter, "\nameref{chap:background}", will provide context for the research by exploring the meaning of web accessibility, the common standards and regulations for achieving it, and the various methods and tools for evaluating it. Since this study focuses on accessibility evaluations in a component library, the final section of this chapter will explain the concept, usage, and importance of component libraries in web development.

The "\nameref{chap:research}" chapter will begin by describing the methodology used in this study, followed by an explanation of the three main components: a survey conducted at Pipedrive to gather insights into the current level of knowledge and awareness around accessibility, an exploratory case study on the usage of automated accessibility testing in Pipedrive's component library, and an overview of a manual accessibility audit on the same component library. The "\nameref{chap:results}" chapter will analyze the findings of this research to understand the capabilities of automated accessibility testing tools and to find answers to the research questions posed earlier.

The final chapter "\nameref{chap:conclusions}", will focus on exploring the limitations of this research and possibilities for future research on similar or related subjects. It will also explain the contributions of this thesis to the research of automated accessibility testing. The chapter will conclude with a summary of the key findings and a reevaluation of the statements proposed at the beginning of this paper.

\subsection{Research Problem and Goal}

% \info{ Research Problem and Significance: What is the problem you are addressing? Why is this a problem? Why is this important to HCI? Supported scientific references? ate 3 4 ½ page max \\
% Research Goal and Motivation: Why are you doing this research? what will be your contributions? Supported scientific references? ate 3 4 ½ page max}

According to \ac{who} 1 billion people or about 16\% of the world's population are estimated to have a significant disability \citep{WHO2022}. They may not be able to use websites and mobile applications in a conventional way. To provide equal possibilities, digital environments need to be made accessible. Focusing on accessibility also has a profound effect on the overall quality of products and services and therefore provides a big business value \citep{Miesenberger2020}.

Developers find it hard to maintain a high level of accessibility over time as it is considered more at the first release of the product \citep{Paterno2020}. \ac{ci} is used to speed up development and maintain general code quality \citep{Zhao2017} and it could also help in maintaining a high level of accessibility by testing the added code against predefined accessibility requirements. There is some initial research into the current state of accessibility testing in \ac{ci}, but it could be improved by doing a case study for an organization \citep{Sane2021, KelseyAdkins2022}.

Design Systems are an increasingly common way to build websites because they help in producing a consistent user experience \citep{Yew2020}. Influencing how these are built is important in the field of HCI because it will allow us to play a role in shaping the future of User Interfaces. Many companies have design systems to help maintain consistency both in what the end users experience and how the code is written. This is a good place to start with establishing a basic level of accessibility. Making changes there will have the widest impact and will build a solid foundation.

Most accessibility checkers are designed to be used for the entire website or page. However, in the case study conducted during this research, automatic accessibility tests were run on the individual components of a component library, as this is a crucial aspect of most Design Systems \citep{Yew2020}. While certain aspects of accessibility, such as the proper usage of headings or the page language, can only be evaluated within the context of the entire page, there is still much to gain from this type of testing.

Parallels can be drawn between isolated component testing and unit tests in software development. The goal of unit tests is to verify that small units of code function properly \citep[p.60]{Humble2010}. Similarly, accessibility testing of components should ensure that this specific component does not have any issues related to accessibility. The next step would be running end-to-end tests on the whole page or application that is made up of these small parts. End-to-end tests should imitate what would happen when end users are using the product. When individual component tests are working well then there should already be fewer issues when you test the whole page.

This thesis will explore the potential pros and cons of accessibility testing in the \ac{ci} of a component library by trying out an automated accessibility evaluation tool in Pipedrive's component library.

\subsection{Research Questions}

To address the research goals of this thesis, stated above, four research questions have been formulated.

	\begin{RQlist}
		\item How good is the knowledge about accessibility standards, tools and best practices in the company before integrating the accessibility testing tool?
		\item What kind of errors can be caught by running automated accessibility tests on a component library?
		\item To what extent can integrating automated testing into a component library's development pipeline help improve its compliance with \ac{wcag}?
		% This might be the most difficult question to answer. There is not enough time to do another manual audit, but I can generate a new automatic report. What can I read into that? If the state is better now is it because of the automated accessibility tool integration or because we also did a manual audit? Maybe I can get some insights from the end questionnaire + reaching out to some devs who have used these tools during that period. Is this a one-time fix and if now then what evidence do I have that it would have a wider impact?
		\item What are the biggest problems of integrating automated accessibility testing into a component library's development workflow?
	\end{RQlist}

The first research question will be addressed by conducting a survey in Pipedrive. The remaining three research questions will be answered by analyzing the results of the case study and manual accessibility audit.

\end{document}
