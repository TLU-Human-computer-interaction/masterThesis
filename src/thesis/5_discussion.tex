% !TeX root = thesis.tex
\documentclass{master_thesis}
\addbibresource{refs.bib}

\begin{document}

\section{Discussion}

\subsection{Limitations of the study}
\subsubsection{Limitations Storybooks addon-a11y}

The accessibility add-on in Storybook analyzes the examples that have been made for the component and unsuitable examples can cause false results. Like Components triggered by a button described before. This is because the initial HTML that the accessibility tests are being run on only has the button and the tests are not being run again after triggering the element. This could potentially be remedied with better examples.

The biggest limitation of this tool currently is that it can only be viewed in Storybook. To see the number of passes and fails you need to open the accessibility tab for each component. I looked into ways of automating this so that the same checks could be run on every change to the library and added to the continuous integration (CI) workflow. In the current version of Storybook, there is no easy way to do this, but it will become much easier in the next major version.

Upgrading our component library to that version would need some extra work to make it compatible, but I have tested out this solution on a test library, and it seems like it would be an improvement. Running the tests in CI would ensure that they are run every time someone makes a change and not only when we choose to. We could also block changes that don't pass the required accessibility checks.

\todo{Reasons for automated check not being effective}

\subsection{Future research}

\begin{enumerate}
	\item Testing storybook 7 and its test runner.
	\item Components that only show a button-play function + test-runner? Would it work?
	\item Adding accessibility tests as a part of library setup - would that have a bigger impact
\end{enumerate}

\todo{AI for accessibility evaluation}

\end{document}