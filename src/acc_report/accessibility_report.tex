\documentclass{thesis_proposal}

\title{Improving web accessibility by integrating continuous evaluation to development workflow}
\author{Mariann Tapfer}
\email{tapferm@gmail.com}

\begin{document}
\maketitle
Volume:  Min 1800, Max 2300 words

\section{Introduction}
The ISO standard for ergonomics of human-system interaction defines usability as `extent to which a system, product or service can be used by specified users to achieve specified goals with effectiveness, efficiency and satisfaction in a specified context of use' \citep{ISO_2019}
% The same standard defines accessibility as the `extent to which products, systems, services, environments and facilities can be used by people from a population with the widest range of user needs, characteristics and capabilities to achieve identified goals in identified contexts of use' \citep{ISO_2019}
`Web accessibility means that websites, tools, and technologies are designed and developed so that people with disabilities can use them' \citep{WAI_introduction_2022}.
It seems logical to think that these terms have a significant overlap. Can we consider accessibility as a part of general usability or should usability be considered as a part of accessibility.
\begin{RQlist}
	\item Is evidence to support the idea that there is a close relation between accessibility and usability?
\end{RQlist}


Both accessibility and usability have their own logic and nessessity and finding a correlation between the value their provide is a good way to find a stronger incentive to invest time and effort in both causes. Research in the field as compared evaluation of accessibility and usability \citep{onay_durdu_perception_2020, aizpurua_exploring_2016, petrie_relationship_2007} to determine the overlap and undestand what is the connection between these two practices. The first thing is to understand what are we comparing. We could look at in depth definitions of usability and accessibility and try to find the overlap or could use automated evaluation tools to get a comparable numerical rating for both. There is a lot of complexity involved in both practices and the definitions and automated testing might not always reveal what real users percieve.
Experts in the field think that accessible sites are more usable for all users \citep{onay_durdu_perception_2020}.


\pagebreak
FOCUSES ON BUND US SHARD PEOPLE.
comparing blind and sighted went the main issues encountered did not have a long overlap (15\%1 this would be specific to reduced in non existent night duty. Could the same thing apple to other modifier).
Genteel hampering people with mounted ability and normal aleility have a different result?
sight
meteoric spectrum negation
IT seems logical to think that accessibility and wealthy are closely related and have a long overlap but there isn't strong evidence to support this there isn't any evidence to the fact that betas Wetherby would somehow. monsters watery for non disabled now.
Do we talk about:
better ux for everyone
better UX for disabled and better UX for others
Better UX making sites also more accessible for disabled
compliance with WCAG Standard making UX better as an added value, side effect?
\citep{WHO_disability_2021}.





\pagebreak
% \nocite{*}
\printbibliography{}

\end{document}