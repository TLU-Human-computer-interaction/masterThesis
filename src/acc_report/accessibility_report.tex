\documentclass{thesis_proposal}

\title{Improving web accessibility by integrating continuous evaluation to development workflow}
\author{Mariann Tapfer}
\email{tapferm@gmail.com}

\begin{document}
\maketitle
Volume:  Min 1800, Max 2300 words

\section{Introduction}
The ISO standard for ergonomics of human-system interaction defines usability as `extent to which a system, product or service can be used by specified users to achieve specified goals with effectiveness, efficiency and satisfaction in a specified context of use' \citep{ISO_2019}. The same standard defines accessibility as the `extent to which products, systems, services, environments and facilities can be used by people from a population with the widest range of user needs, characteristics and capabilities to achieve identified goals in identified contexts of use' \citep{ISO_2019}`. Web Accessibility Initiative that has created and maintained the most used standards in the field defines web accessibility  as means that websites, tools, and technologies are designed and developed so that people with disabilities can use them \citep{WAI_introduction_2022}.
They all deal with making sure that the end-users of websites can achieve their goals on them with ease, accessibility also focuses on people with reduced abilities.
It seems logical to think that these terms have a significant overlap. If someone with lesser abilities can use a site then we could assume that anyone else should not have any difficulty managing. Can we consider accessibility as a part of general usability or should usability be considered as a part of accessibility.
\begin{RQlist}
	\item How are usability and accessibility of web content related?
\end{RQlist}

\section{Why is an overlap between usability and accessibility relevant}

Both accessibility and usability have their own logic and necessity, finding a correlation between the value their provide is a good way to find a stronger incentive to invest time and effort in both causes. High correlation between the two could be a good extra argument to invest time in dealing with them both.
in addition Aizpurua came to the conclusion that a website that is experienced to be accessible is perceived as good, appealing and beautiful, while a non-accessible Web site is considered as bad repelling and ugly \citep{aizpurua_exploring_2016}.
There is a growing number of articles written about accessibility and usability of government websites. This is probably related to laws in different countries making following accessibility standards compulsory and accessing government data trough web becoming more popular.

\section{How can we compare usability and accessibility}
The first thing is to understand what are we comparing. We could look at in depth definitions of usability and accessibility and try to find the overlap or could use automated evaluation tools to get a comparable numerical rating for both. There is a lot of complexity involved in both practices and the definitions and automated testing might not always reveal what real users might perceive.\\
The biggest difficulty in definitively determining the potential correlation between usability and accessibility starts from how both of them are measures and evaluated on their own. There are some automated solutions, standards and checklists, but in most cases it is still recommended to do some additional testing with real users to get the best results. A website might follow  WCAG standard at a sufficient level, but still be inaccessible to users \citep{aizpurua_exploring_2016}. Dealing with human perception means it can differ from person to person based on their past experience and level of abilities.\\

\section{User testing to see determine correlation between usability and accessibility}
There is research in the field that has compared evaluation of accessibility and usability \citep{onay_durdu_perception_2020, aizpurua_exploring_2016, petrie_relationship_2007} to determine the overlap and understand what is the connection between these two practices.
Most studies directly comparing usability and accessibility have been run with blind people in the non disabled group. This could be an unfair comparison even tough this group of disabled users in the one that face most difficulty because web is designed to be very visual \citep{aizpurua_exploring_2016}.
Disability is a general umbrella term that includes a wide range of people with varied abilities in different areas. In terms of usability we mostly talk about people with abilities that are considered normal or regular even tough there is also some variation in there the differences within the disabled population are infinitely bigger. It can varied from people with very bad eyesight to the blind and even users who can see, hear or move their limbs in a  regular way. Making generalisations based on one group can lead to wrong conclusions. There has been research to determine some solid conclusions on certain groups of users. Research suggests that higher compliance with WCAG guidelines makes sites more usable for older people and also improve user experience for functionally illiterate individuals\citep{exploring_2018}. The latter would probably also assume that sufficient attention has been put on the content quality and compliance with accessibility standards.
There are certain aspects of accessibility and usability that have been proven to always have a high correlation like being visually clean \citep{exploring_2018}.
We could say that usability issues for disabled users start when a website has the minimum level that allows them to perceive the information. The next step would be to provide that content in a way that is efficient and satisfying to use.

\section{Perception of the correlation between usability and accessibility}
There have been studies looking at perception – how experts and non experts perceive this correlation. Even tough this does not prove that there is in fact a significant relationship between usability and accessibility is it still relevant to mention because these experts are the ones doing the most work in this field and therefore have a very direct impact on how it will evolve in the future.\\
Experts in the field think that accessible sites are more usable for all users, web accessibility problems affect all users regardless of their situational of physical limitations and even prefer more inclusive definitions of accessibility \citep{onay_durdu_perception_2020}.

\section{Conclusion}
From this we can see that even if the evidence of the correlation does not yet definitively prove its significance in every aspect and variation most practitioners and experts see these two concepts as closely related entities already. This could potentially have a very positive side affect - while working on solving accessibility issues practitioners might already consider usability principles and vice versa. This would mean that the end result would provide a better user experience and would not exclude big groups of users.

\pagebreak
% \nocite{*}
\printbibliography{}

\end{document}